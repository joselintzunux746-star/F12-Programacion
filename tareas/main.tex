\documentclass[12pt]{article}

% --- Página y tipografía ---
\usepackage[letterpaper,margin=2.5cm]{geometry}
\usepackage[T1]{fontenc}
\usepackage[utf8]{inputenc} % si compilas con pdfLaTeX
\usepackage{lmodern}
\usepackage{microtype}

% --- Imágenes y color ---
\usepackage{graphicx}
\usepackage{xcolor}

% --- Control fino de espacios ---
\usepackage{setspace}
\setlength{\parindent}{0pt}

\begin{document}
\thispagestyle{empty}

% ===== Encabezado con logos + texto =====
\begin{minipage}[c]{0.18\textwidth}
    \centering
    % Cambia por tu logo izquierdo
    \includegraphics[width=0.95\linewidth]{img/logo_usac.jpeg}
\end{minipage}
\hfill
\begin{minipage}[c]{0.60\textwidth}
    \small
    Universidad de San Carlos de Guatemala\\
    Escuela de Ciencias Físicas y Matemáticas\\
    Nombre estudiante: Joselin Liseth Tzunux Coxaj\\
    Carnet: 202606884 \\
    Programación 1\\
\end{minipage}
\hfill
\begin{minipage}[c]{0.18\textwidth}
    \centering
    % Cambia por tu logo derecho
    \includegraphics[width=1.4\linewidth]{img/logo_ecfm.jpg}
\end{minipage}

\vspace{0.5cm}

% Línea horizontal superior (gruesa)
\noindent\rule{\textwidth}{1.2pt}

\vspace{0.2cm}

% ===== Título =====
\begin{center}
    {\Large\scshape tarea 1	}\\[0.3em]
\end{center}

\vspace{0.1cm}

% Fecha
\begin{center}
    \small\scshape 05 de febrero de 2026
\end{center}

\vspace{0.2cm}

% Línea horizontal inferior (gruesa)
\noindent\rule{\textwidth}{1.2pt}

\vspace{0.6cm}

% ===== Caja de resumen =====
\noindent
\colorbox{gray!35}{%
    \parbox{\textwidth}{%
        \vspace{0.6em}
        \textbf{Ensayo}\\[0.3em]
        \small

        \vspace{0.8em}
    }%
}


\section*{¿Cuál sería mí área de investigación en Física? }


La Física es una de las ciencias fundamentales para comprender el universo. A lo largo de la historia, ha permitido explicar desde los fenómenos más simple del movimiento hasta los comportamientos más complejos del espacio y el tiempo. Dentro de este amplio campo, una de las áreas de investigación más fascinantes y profundas es la teoría de la relatividad, propuesta por Albert Einstein a inicios del siglo XX. Esta teoría no solo revolucionó la manera en que entendemos el tiempo, el espacio y la gravedad, sino que también abrió nuevas puertas al estudio del cosmos, ya que solíamos pensar a menudo que el tiempo es como un rio, que fluye a la misma velocidad para todos, sin embargo, descubrir que el tiempo puede estirarse o encogerse, cambio por completo mi percepción de la realidad. Mi fascinación por la Física y específicamente en el área de la relatividad, dándome cuenta que es mucho mas extraño y elegante de lo que nosotros podemos percibir.  En la actualidad, la relatividad sigue siendo un pilar fundamental en la investigación científica, y su desarrollo y aplicación dependen en gran medida del uso de la programación.\\

Investigar acerca de la relatividad es importante porque nos permite entender el funcionamiento del universo a gran escala. Gracias a esta teoría, los científicos pueden estudiar el origen del cosmos, la evolución de las galaxias y los eventos extremos que ocurren en el espacio profundo, al igual nos ayuda a comprender mejor los fenómenos como los agujeros negros, las ondas gravitacionales y la expansión del universo. Además, la relatividad no es solo un tema teórico, tiene aplicaciones prácticas en la vida cotidiana, un ejemplo claro es el sistema de posicionamiento global (GPS), que necesita correcciones relativistas para ofrecer ubicaciones precisas. Sin estas correcciones, los errores en la medición del tiempo causarían fallas significativas en la localización.\\

En este contexto, la programación juega un papel esencial en la investigación en física moderna, ya los fenómenos descritos por la relatividad suelen involucrar ecuaciones matemáticas complejas que no pueden resolverse fácilmente de forma analítica. Por esta razón, los físicos utilizan programas de computadora para simular sistemas, analizar grandes cantidades de datos y resolver modelos matemáticos. Mediante lenguajes de programación como Python, C++ o MATLAB, en donde se puedan crear simulaciones de la curvatura del espacio-tiempo, el movimiento de objetos en campos gravitatorios intensos o la propagación de ondas gravitacionales. Además, la programación permite trabajar con datos obtenidos de telescopios, satélites y detectores especializados. De esta manera, la programación se convierte en un puente entre la teoría física y la observación experimental.\\


La relatividad no es para mí un simple conjunto de ecuaciones, si no el lenguaje que describe la composición de la realidad. Al mismo tiempo, la programación es una herramienta indispensable para desarrollar esta área, ya que facilita el análisis, la simulación y la interpretación de datos complejos. La combinación de física teórica y habilidades computacionales no solo impulsa el avance del conocimiento científico, sino que también prepara a los investigadores para enfrentar los retos tecnológicos y científicos del futuro. 
\end{document}

\end{document}
